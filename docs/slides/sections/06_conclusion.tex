% Section 6: Conclusion

\section{Conclusion}

\begin{frame}{What Was Achieved}
    \begin{block}{Successfully Implemented}
        \begin{itemize}
            \item \faCheckCircle\ \textbf{Async queue-based execution} with Redis + BullMQ
            \item \faCheckCircle\ \textbf{Multi-language support} (JavaScript, Python, Java)
            \item \faCheckCircle\ \textbf{Session management} with autosave and throttling
            \item \faCheckCircle\ \textbf{Code isolation} with timeout and output limits
            \item \faCheckCircle\ \textbf{Idempotency protection} with 4-layer defense
            \item \faCheckCircle\ \textbf{Automatic failure recovery} via BullMQ retry logic
            \item \faCheckCircle\ \textbf{Scalable worker architecture} with horizontal scaling
        \end{itemize}
    \end{block}
\end{frame}

\begin{frame}{Production Readiness Gaps}
    \begin{alertblock}{What Would Be Improved with More Time}
        \textbf{1. Stronger Container Isolation} \\
        {\footnotesize Replace child processes with Docker or gVisor for full sandboxing}
        
        \vspace{0.2cm}
        
        \textbf{2. PostgreSQL Migration} \\
        {\footnotesize Enable concurrent writes, read replicas, and horizontal API scaling}
        
        \vspace{0.2cm}
        
        \textbf{3. Observability} \\
        {\footnotesize Structured logging, Prometheus metrics, BullMQ dashboard, alerting}
        
        \vspace{0.2cm}
        
        \textbf{4. Real-time Updates} \\
        {\footnotesize WebSocket/SSE for pushing execution status instead of polling}
        
        \vspace{0.2cm}
        
        \textbf{5. Advanced Queue Management} \\
        {\footnotesize Max queue size, priority queues, per-language isolation, backpressure}
    \end{alertblock}
\end{frame}

% \begin{frame}{Key Technical Lessons}
%     \begin{block}{System Design Thinking}
%         \begin{enumerate}
%             \item \textbf{Decouple components for independent scaling} \\
%             {\footnotesize Queue-based architecture allows API and Workers to scale separately}
            
%             \vspace{0.2cm}
            
%             \item \textbf{Database as single source of truth} \\
%             {\footnotesize Avoid distributed state synchronization complexity}
            
%             \vspace{0.2cm}
            
%             \item \textbf{Defense in depth for critical operations} \\
%             {\footnotesize Multiple layers prevent single point of failure (idempotency, rate limiting)}
            
%             \vspace{0.2cm}
            
%             \item \textbf{Trade-offs are conscious decisions} \\
%             {\footnotesize Every choice (SQLite, polling, child process) has documented rationale}
            
%             \vspace{0.2cm}
            
%             \item \textbf{Design for current needs, plan for growth} \\
%             {\footnotesize Simple now, clear migration path to production scale}
%         \end{enumerate}
%     \end{block}
% \end{frame}

\begin{frame}{Thank You}
    \begin{center}
        \Huge Live Code Execution Backend
        
        \vspace{0.5cm}
        
        \Large A Case Study in Scalable System Design
        
        \vspace{1cm}
        
        \normalsize
        \textbf{GitHub:} \texttt{github.com/BuhDuy256/live-code-execution-backend}
        
        \vspace{0.3cm}
        
        \textbf{Tech Stack:} Node.js, TypeScript, Redis, BullMQ, SQLite
        
        \vspace{1cm}
        
        \large \textit{Questions?}
    \end{center}
\end{frame}
